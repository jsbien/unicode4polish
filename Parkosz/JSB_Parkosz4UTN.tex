% xelatex -shell-escape JSB_Parkosz4UTN.tex
% Janusz S. Bień
% Formal Linguistics Department, University of Warsaw, Dobra 55, 00-312 Warszawa, Poland
% jsbien@uw.edu.pl
\documentclass{mwart}
\usepackage{fontspec}
\usepackage{polyglossia}
\setmainlanguage{english}
\setotherlanguage{polish}
\usepackage{csquotes}
\usepackage[style=authoryear,backend=biber]{biblatex}
%\addbibresource{jsbqed.bib}
\addbibresource{paleografia.bib}
\addbibresource{JSB2013.bib}
\addbibresource{zs_unibipod.bib}
\addbibresource{jsbquoted.bib}
\addbibresource{Parkosz.bib}

% http://tex.stackexchange.com/questions/166337/quotation-mark-quotation-sign-xelatex-polyglossia-csquotes
\DeclareQuoteStyle{polish}% I looked it up on Wikipedia, no idea if it's right
  {\quotedblbase}
  {\textquotedblright}
  [0.05em]
  {\textquoteleft}
  {\textquoteright}



\setmainfont[Mapping=tex-text]{TeX Gyre Termes}
% \char"EC10 \char"EC11 oraz \char"EC12
\def\orogate{\char"EC12}
\usepackage{relsize}

\usepackage{hyperref}

\usepackage{graphicx}
% [hyphens]: options clash
\usepackage{url}
%\usepackage{natbib}
\newcommand{\uname}[1]{{\relsize{-1}\texttt{#1}}}
\newcommand{\ucode}[1]{\texttt{#1}}
\newcommand{\mname}[1]{\texttt{#1}}
\newcommand{\mcode}[1]{\texttt{#1}}
\newcommand{\aname}[1]{\texttt{#1}}
\newcommand{\acode}[1]{\texttt{#1}}
\newcommand{\prname}[1]{\textsf{#1}}
%


\usepackage{draftwatermark}

% http://tex.stackexchange.com/questions/184672/embedding-mercurial-version-control-information-in-a-tex-document

\begin{document}

% 
\title{Transcribing old Polish in Unicode.\\ A case study of a 15th century treatise}

\author{Janusz S. Bień}


\date{\today}

\maketitle

\begin{center}

Mercurial changeset and local revision:

\input{"| hg log -v -l 1 \jobname.tex --template '{node}'"}

\input{"| hg log -v -l 1 \jobname.tex --template '{rev}'"}

\end{center}
\begin{abstract}

  At the moment the present document consists just of some loose notes
  about representing in Unicode the treatise by Jakub Parkosz alias
  Parkoszowic and its critical editions. It is hoped that it will
  evaluate towards a Unicode Technical Notes.

  The present author would be happy to discuss the presented ideas on
  any suitable forum and would welcome any contribution to the present
  text.
\end{abstract}

\section{Introduction}
\label{sec:introduction}

The starting point for the present work are various papers and
presentations in Polish including \url{http://bc.klf.uw.edu.pl/469/}
and the results of experiments available at
\url{https://bitbucket.org/jsbien/parkosz-font/} and
\url{https://bitbucket.org/jsbien/parkosz-traktat}.

\section{Parkosz’s orthographic treatise}
\label{sec:park-orth-treat}

\subsection{The manuscript}
\label{sec:manuscript}

\subsection{The first printed edition}
\label{sec:first-print-edit}

\subsection{The second printed edition}
\label{sec:second-print-edit}

\subsection{The third printed edition}
\label{sec:third-print-edit}

\subsection{The electronic edition under consideration}
\label{sec:electr-edit-under}



%\printbibliography
\end{document}


  In 1985 Prof. Kucała published a book on Parkosz’s orthographic
  treatise
  containing the facsimile of the manuscript, its Latin text
  and the polish translation. Later Prof. Kucała agreed to apply to
  the book the Creative Commons Attribution license. 

auctex: "TeX-command-extra-option" has no effect
https://bugs.debian.org/cgi-bin/bugreport.cgi?bug=824945

%%% Local Variables: 
%%% coding: utf-8-unix
%%% mode: latex
%%% TeX-master: t
%%% TeX-PDF-mode: t
%%% TeX-engine: xetex
%%% TeX-command-extra-options: "-shell-escape"
%%% End: 
