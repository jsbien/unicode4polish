% https://apcz.umk.pl/czasopisma/index.php/sztukaedycji/index
% https://apcz.umk.pl/czasopisma/index.php/sztukaedycji/about/submissions#authorGuidelines
% Janusz S. Bień
% Formal Linguistics Department, University of Warsaw, Dobra 55, 00-312 Warszawa, Poland
% jsbien@uw.edu.pl

\documentclass{mwart}
\usepackage{fontspec}
\usepackage{polyglossia}
\usepackage{etoolbox}
\setmainlanguage{english}
%\setotherlanguage{english}
\usepackage{csquotes}
\usepackage{wrapfig}
\usepackage[style=authoryear,backend=biber,backref=true,urldate=short]{biblatex}
\addbibresource{all.bib}

% http://tex.stackexchange.com/questions/166337/quotation-mark-quotation-sign-xelatex-polyglossia-csquotes
\DeclareQuoteStyle{polish}% I looked it up on Wikipedia, no idea if it's right
  {\quotedblbase}
  {\textquotedblright}
  [0.05em]
  {\textquoteleft}
  {\textquoteright}

% psuje tytuły???:
%  \usepackage{ulem}
\usepackage{metalogo}
\usepackage[polish]{varioref}
\usepackage{xcolor}


\def\eob{ę}

\setmainfont[Mapping=tex-text]{TeX Gyre Termes}
% \char"EC10 \char"EC11 oraz \char"EC12
\def\orogate{\char"EC12}
\def\medievalcomma{{\fontspec{Unifont}⹌}}
\def\Hb#1{{\fontspec{Junicode}#1}}
\def\Htest{{\fontspec{Unifont}M⁹¹}}
% \newcommand{\J}[1]{{\fontspec[Path=/home/jsbien/Junicode/]{Junicode.ttf}#1}}
% \def\Sł{{\fontspec[Path=/home/jsbien/Junicode/]{Junicode.ttf}\char"E8DF}}

\newfontfamily\J[Color=green]{JuniusX}
\newfontfamily\bJ[Color=violet]{Junicode}
%\newcommand{\J}[1]{{\bJ#1}}
\def\Sł{{\fontspec{Junicode.ttf}\char"E8DF}}
% buster: Missing character: There is no  in font [Junicode.ttf]/OT:script=latn;language
\newfontfamily\bS[Color=violet]{Symbola}
\newcommand{\Sy}[1]{{\bS#1}}
\newcommand{\Ju}[1]{{\bJ#1}}

\usepackage{relsize}

\usepackage{hyperref}

\usepackage{graphicx}
% [hyphens]: options clash
\usepackage{url}
%\usepackage{natbib}

% program name
\newcommand{\pname}[1]{\textsf{#1}}


% file name
\newcommand{\fname}[1]{\texttt{#1}}

\newcommand{\uname}[1]{\texttt{'#1'}}
\newcommand{\ucode}[1]{\texttt{U+#1}}
\newcommand{\usi}[1]{\texttt{#1}}

% Aletheia
\newcommand{\aname}[1]{\texttt{#1}}
\newcommand{\acode}[1]{\texttt{#1}}

% MUFI
\newcommand{\mname}[1]{\texttt{'#1 \textsc{<mufi>'}}}
\newcommand{\mcode}[1]{\texttt{M+#1}}



\usepackage{draftwatermark}
\usepackage[doublespacing]{setspace}

\usepackage{xfrac}

% ni edziała:?
\renewcommand{\topfraction}{0.9}
\renewcommand{\floatpagefraction}{0.9}	% require fuller float pages

\newcommand{\Jglyph}[1]{{\relsize{2}\J#1}}

%\gappto\captionslingua{\renewcommand{\chaptername}{Caput}}
%\gappto\captionspolish{\renewcommand{\figurename}{Ilustracja}}


% \newcommand{\uname}[1]{{\relsize{-1}\texttt{#1}}}
% \newcommand{\ucode}[1]{\texttt{#1}}
% \newcommand{\mname}[1]{\texttt{#1}}
% \newcommand{\mcode}[1]{\texttt{#1}}
% \newcommand{\aname}[1]{\texttt{#1}}
% \newcommand{\acode}[1]{\texttt{#1}}
% \newcommand{\prname}[1]{\textsf{#1}}

\usepackage[mark]{gitinfo2}

\begin{document}

% 
%\title{Elektroniczna typemiczna transkrypcja traktatu Stanisława Zaborowskiego}
\title{Textel names proposal for JuniusX\\ Unicode plane 15 characters}

\author{Janusz S. Bień}

\date{\today\\\gitAuthorIsoDate}

\maketitle

\catcode`\&=11
\catcode`\|=11
\catcode`\_=11

% The notion of textel was proposed ???

% JuniusX ???

% Unicode

The names are to be used in particular with \textsf{Unihistext}
(\url{https://bitbucket.org/jsbien/unihistext}).  The characters
themselves are to be used in particular with \textsf{djview4poliqarp}
(\url{https://bitbucket.org/mrudolf/djview-poliqarp/}), which does not
support OpenType features, for indexes such as
\url{https://github.com/jsbien/Zaborowski-index4djview}.

Some principles of the naming policy:
\begin{itemize}
\item Comments in brackets are considered as a part of the name by \textsf{Unihistext}, but not necessarily in other circumstances.
\item Every proper name ends in VARIANT
\item Names try to follow English Unicode usage. The term
  \textit{terminal} comes from
  \autocite{gaskell76:_nomec_letter_roman_type}.
% \item ; when appropriate, MUFi names are used or adapted. MUFI codes
%   in comments are prefixed with M+.

\end{itemize}

\begin{description}
\item[0xF0000] LATIN CAPITAL LETTER A WITH STROKE THROUGH TERMINAL
  VARIANT [JuniusX]: \Jglyph{󰀀}.\\
  Cf. \url{https://github.com/psb1558/Junicode-New/issues/14}.
  % The
  % term \textit{terminal} used after ???.\\
  
  In JuniusX font accessible also as \textit{A} with \texttt{cv02[1]};
  a slightly different glyph available as \texttt{cv02[2]},
  cf. \autocite[p. 7]{baker20:_opent_featur_junius_junius}.
% 
%  Used in particular in ???, cf. Fig. ???

\item [0xF0001] LATIN SMALL LETTER A WITH STROKE THROUGH TERMINAL VARIANT [JuniusX]: 
  \Jglyph{󰀁}.\\ Cf. \url{https://github.com/psb1558/Junicode-New/issues/14}.
  % The
  % term \textit{terminal} used after ???.

  In JuniusX font accessible also as \textit{A} with \texttt{cv02[1]};
  a slightly different glyph available as \texttt{cv02[2]},
  cf. \autocite[p. 7]{baker20:_opent_featur_junius_junius}.


\item [0xF0002] LATIN SMALL LETTER A WITH BENT STROKE THROUGH TERMINAL VARIANT [JuniusX]: 
  \Jglyph{󰀂}.\\ Cf. \url{https://github.com/psb1558/Junicode-New/issues/14}.
  % The
  % term \textit{terminal} used after ???.

  In JuniusX font accessible also as \textit{A} with \texttt{cv02[3]};
  cf. \autocite[p. 7]{baker20:_opent_featur_junius_junius}.


\item [0xF0003] LATIN SMALL LETTER C WITH LOW OVERLINE VARIANT [JuniusX]: 
  \Jglyph{󰀃}.\\ Cf. \url{https://github.com/psb1558/Junicode-New/discussions/44#discussioncomment-202860}.
\item [0xF0004] LATIN SMALL LETTER C WITH HIGH OVERLINE VARIANT [JuniusX]: 
  \Jglyph{󰀄}.\\ Cf. \url{https://github.com/psb1558/Junicode-New/discussions/44#discussioncomment-202860}.
\item [0xF0005] LATIN SMALL LETTER INSULAR D VARIANT [JuniusX]:\\
  \Jglyph{󰀅}.%\\ %Cf. \url{https://github.com/psb1558/Junicode-New/issues/14}.

  In JuniusX font accessible also as \textit{d} with \texttt{cv05[2]};
  cf. \autocite[p. 7]{baker20:_opent_featur_junius_junius}.

  May be considered also as a variant of LATIN SMALL LETTER INSULAR D (U+A77A).


\item [0xF0006] LATIN SMALL LETTER ETH VARIANT [JuniusX]: 
  \Jglyph{󰀆}.\\ Cf. \url{https://github.com/psb1558/Junicode-New/discussions/44#discussioncomment-202860}.

  In JuniusX font accessible also as \textit{ð} (U+00F0) with \texttt{ss01};
  cf. \autocite[p. 11]{baker20:_opent_featur_junius_junius}.

\item [0xF0007] LATIN SMALL LETTER D WITH STROKE ROUNDED VARIANT [JuniusX]:\\
  \Jglyph{󰀇}.\\ Cf. \url{https://github.com/psb1558/Junicode-New/issues/??}.

    In JuniusX font accessible also as \textit{đ}  (U+0111) with \texttt{cv06[1]};
  cf. \autocite[p. 7]{baker20:_opent_featur_junius_junius}.

  
\item [0xF0008] LATIN SMALL LETTER D WITH HIGH OVERLINE VARIANT:\\
  \Jglyph{󰀈}.\\ Cf. \url{https://github.com/psb1558/Junicode-New/discussions/44#discussioncomment-202860}.
\item [0xF0009] LATIN CAPITAL LETTER E WITH STROKE THROUGH TERMINAL VARIANT [JuniusX]:\\
  \Jglyph{󰀉}.\\ Cf. \url{https://github.com/psb1558/Junicode-New/issues/13}.
\item [0xF000A] LATIN SMALL LETTER E WITH STROKE THROUGH TERMINAL VARIANT [JuniusX]:\\
  \Jglyph{󰀊}.\\ Cf. \url{https://github.com/psb1558/Junicode-New/issues/13}.
\item [0xF000B] LATIN SMALL LETTER F VARIANT [JuniusX]:\\
  \Jglyph{󰀋}.\\ % Cf. \url{https://github.com/psb1558/Junicode-New/issues/??}.
  In JuniusX font accessible also as \textit{f} with \texttt{cv09[5]};
  cf. \autocite[p. 9]{baker20:_opent_featur_junius_junius}.

\item [0xF000C] LATIN SMALL LETTER I WITH HIGH OVERLINE VARIANT [JuniusX]:\\
  \Jglyph{󰀌}.\\ Cf. \url{https://github.com/psb1558/Junicode-New/discussions/44#discussioncomment-202860}.
\item [0xF000D] LATIN SMALL LETTER J VARIANT [JuniusX]:\\
  \Jglyph{󰀍}.\\ % Cf. \url{https://github.com/psb1558/Junicode-New/issues/??}.

  % In JuniusX font accessible also as \textit{đ}  (U+0111) with \texttt{cv06[1]};
  % cf. \autocite[p. 7]{baker20:_opent_featur_junius_junius}.
\item [0xF000E] LATIN SMALL LETTER J WITH HIGH OVERLINE  VARIANT [JuniusX]:\\
  \Jglyph{󰀎}.\\ Cf. \url{https://github.com/psb1558/Junicode-New/discussions/44#discussioncomment-202860}.
\item [0xF000F] LATIN SMALL LETTER L WITH HIGH ROUNDED STROKE VARIANT [JuniusX]:\\
  \Jglyph{󰀏}.\\  Cf. \url{https://github.com/psb1558/Junicode-New/issues/4}.
  In JuniusX font accessible also as \textit{ꝉ} (U+A749) with \texttt{cv17[1]};
  cf. \autocite[p. 9]{baker20:_opent_featur_junius_junius}.

\item [0xF0010] LATIN SMALL LETTER M WITH HIGH OVERLINE VARIANT [JuniusX]:\\
  \Jglyph{󰀐}.\\ Cf. \url{https://github.com/psb1558/Junicode-New/discussions/44#discussioncomment-202860}.
 \item [0xF0011] LATIN CAPITAL LETTER O WITH STROKE HORNED VARIANT [JuniusX]\\
  \Jglyph{󰀀}.\\  Cf. \url{https://github.com/psb1558/Junicode-New/issues/3}.
  In JuniusX font accessible also as \textit{ø} (U+00F8) with \texttt{cv21[1]};
  cf. \autocite[p. 9]{baker20:_opent_featur_junius_junius}.
\item [0xF0012] LATIN CAPITAL LETTER O WITH SHIFTED STROKE VARIANT [JuniusX]\\:
  \Jglyph{󰀒}.\\  Cf. \url{https://github.com/psb1558/Junicode-New/issues/3}.

  In JuniusX font accessible also as \textit{ø} (U+00F8) with \texttt{cv21[2]};
  cf. \autocite[p. 9]{baker20:_opent_featur_junius_junius}.
\item [0xF0013] LATIN CAPITAL LETTER O WITH STROKE HORN DOWNWARDS VARIANT [JuniusX]\\:
  \Jglyph{󰀓}.\\  Cf. \url{https://github.com/psb1558/Junicode-New/issues/3}.

  In JuniusX font accessible also as \textit{ø} (U+00F8) with \texttt{cv21[3]};
  cf. \autocite[p. 9]{baker20:_opent_featur_junius_junius}.
\item [0xF0014] LATIN CAPITAL LETTER O WITH STROKE HORN UPWARDS VARIANT [JuniusX]\\:
  \Jglyph{󰀓}.\\  Cf. \url{https://github.com/psb1558/Junicode-New/issues/3}.

  In JuniusX font accessible also as \textit{ø} (U+00F8) with \texttt{cv21[4]};
  cf. \autocite[p. 9]{baker20:_opent_featur_junius_junius}.
\item [0xF0015] ???  [JuniusX]\\:
  \Jglyph{󰀕}.\\%  Cf. \url{https://github.com/psb1558/Junicode-New/issues/3}.
\item [0xF0016] LATIN SMALL LETTER RUM ROTUNDA VARIANT [JuniusX]\\:
  \Jglyph{󰀖}.\\%  Cf. \url{https://github.com/psb1558/Junicode-New/issues/3}.

  In JuniusX font accessible also as \textit{ꝝ} (U+A75D) with \texttt{cv41[1]};
  cf. \autocite[p. 12]{baker20:_opent_featur_junius_junius}.
\item [0xF0017] LATIN SMALL LETTER V WITH LOW OVERLINE VARIANT [JuniusX]:// 
  \Jglyph{󰀗}.\\ Cf. \url{https://github.com/psb1558/Junicode-New/discussions/44#discussioncomment-202860}.
\item [0xF0018] LATIN SMALL LETTER V WITH HIGH OVERLINE VARIANT [JuniusX]:// 
  \Jglyph{󰀘}.\\ Cf. \url{https://github.com/psb1558/Junicode-New/discussions/44#discussioncomment-202860}.
\item [0xF0019] LATIN SMALL LETTER X WITH LOW OVERLINE VARIANT [JuniusX]:// 
  \Jglyph{󰀙}.\\ Cf. \url{https://github.com/psb1558/Junicode-New/discussions/44#discussioncomment-202860}.
\item [0xF001A] LATIN SMALL LETTER V WITH HIGH OVERLINE VARIANT [JuniusX]:// 
  \Jglyph{󰀚}.\\ Cf. \url{https://github.com/psb1558/Junicode-New/discussions/44#discussioncomment-202860}.
\item [0xF001B] LATIN LETTER GLOTTAL STOP VARIANT [JuniusX]\\
  \Jglyph{󰀛}.\\%  Cf. \url{https://github.com/psb1558/Junicode-New/issues/??}.
% \\item [0xF001C] LATIN SMALL LETTER RUM ROTUNDA VARIANT [JuniusX]\\:
\item [0xF001C] TIRONIAN SIGN CAPITAL ET VARIANT [JuniusX]\\:
\Jglyph{󰀜}.%\\%  Cf. \url{https://github.com/psb1558/Junicode-New/issues/??}.

In JuniusX font accessible also as \textit{⹒} (U+2E52) with \texttt{cv40[1]};
  cf. \autocite[p. 12]{baker20:_opent_featur_junius_junius}.

\item [0xF001D] TIRONIAN SIGN ET VARIANT [JuniusX]\\:
  \Jglyph{󰀝}.\\%  Cf. \url{https://github.com/psb1558/Junicode-New/issues/??}.
  In JuniusX font accessible also as \textit{⁊} (U+2E52) with \texttt{cv40[1]};
  cf. \autocite[p. 12]{baker20:_opent_featur_junius_junius}.
\end{description}

\printbibliography
\end{document}

%%% Local Variables: 
%%% coding: utf-8-unix
%%% eval: (set-fontset-font "fontset-default" '(#xF0000 . #xF000A) (font-spec :size 18 :name "JuniusX"))
%%% mode: latex
%%% TeX-master: t
%%% TeX-PDF-mode: t
%%% TeX-engine: xetex
%%% End: 
