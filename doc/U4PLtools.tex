\documentclass[12]{mwart}
% \usepackage{polyglossia}
% \setdefaultlanguage{polish}

\usepackage[no-math]{fontspec}

\usepackage{enumitem}
\usepackage{draftwatermark}

\usepackage{xltxtra}

\setmainfont[Mapping=tex-text]{TeX Gyre Termes}
\setsansfont[Mapping=tex-text]{TeX Gyre Adventor}
%\setmainfont{TeXGyreTermes}
%\setmainfont{DejaVu Serif}
%\setmainfont{Bitstream Vera Serif}
%\setmonofont{TeX Gyre Cursor}
\setmonofont{DejaVu Sans Mono}
\usepackage{draftwatermark}

% \usepackage{bibentry,natbib}

\usepackage{graphicx}

\usepackage{hyperref}

\usepackage{soul}

\usepackage{relsize}

\usepackage[style=authoryear,natbib=true]{biblatex}
%\addbibresource{JSB2013.bib,typografia.bib}
%\addbibresource{4JSB2014.bib}
\AtEveryBibitem{\clearfield{note}}



\newcommand{\program}[1]{\textsf{#1}}

\title{U4PL tools}
\author{Janusz S. Bień}

\date{\today}


\begin{document}
\maketitle
% \pagestyle{empty}

% no math
\catcode`\&=12
\catcode`\_=12

\section{Unicode Character Database}
\label{sec:unic-char-datab}

On Debian the tools use by default DCB provided by the package
\texttt{unicode-data}. The location of DCB should be however
configurable for portability and for access to earlier versions of the
standards.

\section{Wiki structure}
\label{sec:wiki-structure}

It is yet defined definitively.

\section{U4PL file indexer}
\label{sec:u4pl-file-indexer}

The program scans recursively the \texttt{textels} subdirectory for the
\texttt{rst} files. For every file its title is extracted,
cf. \url{https://mercurial.selenic.com/wiki/HelpOnParsers/ReStructuredText/RstPrimer}:
\begin{quote}
  The title of the whole document is distinct from section titles and
  may be formatted somewhat differently (e.g. the HTML writer by
  default shows it as a centered heading).

  To indicate the document title in reStructuredText, use a unique
  adornment style at the beginning of the document.
\end{quote}
If a file doesn't contain a title, a warning is issued and the title
ios assumed to be the name of the file with its path relative to
\texttt{textels}.

% Extracted titles can be cashed somewhere to speed up next runs of the
% program.

The titles are used to create entries in the \texttt{Home.rst} file in
the current directory. The entries have the form of hyperlinks such as

{\relsize{-2}
\begin{verbatim}
`TITLE <path/file_name>`_
\end{verbatim}
}

By default the order of entries should reflect the directory
structure; in the future different additional arrangement may appear
useful.



\section{U4PL template creator}
\label{sec:u4pl-templ-creat}

The program takes a hexadecimal code point as an argument. It creates
a template file for it and, if appropriate, the template for the
opposite case with the appropriate hyperlinks.

For example, for \texttt{0253} it creates the file
\texttt{U+0253_LATIN_SMALL_LETTER_B_WITH_HOOK.rst} containing in
particular:

\begin{itemize}
\item \texttt{`'LATIN SMALL LETTER B WITH HOOK' (U+0253) <http://www.fileformat.info/info/unicode/char/0253/index.htm>`_
}
\item \texttt{Unicode 1.1.0 (June, 1993)}
from \texttt{DerivedAge.txt}
\item 
  \begin{quote}
    \obeylines
    	* implosive bilabial stop
	* Pan-Nigerian alphabet
	* uppercase is 0181
  \end{quote}
from \texttt{NamesList.txt}
\item Uppre case \texttt{U+0181} from \texttt{UnicodeData.txt}
\item Wikpedia link \url{http://en.wikipedia.org/wiki/%C9%93} (the last element is in UTF-8, the address may be invalid).
\item DecodeUnicode link \url{http://www.decodeunicode.org/u+0253}
\item The character itself: \texttt{Local font: ɓ}


\end{itemize}

\end{document}

/usr/share/doc/unicode-data

/usr/share/unicode

%%% Local Variables:
%%% mode: latex
%%% TeX-master: t
%%% TeX-engine: xetex
%%% TeX-PDF-mode: t
%%% coding: utf-8
%%% End:
