\documentclass[12]{mwart}
% \usepackage{polyglossia}
% \setdefaultlanguage{polish}

\usepackage[no-math]{fontspec}

\usepackage{enumitem}
\usepackage{draftwatermark}

\usepackage{xltxtra}

\setmainfont[Mapping=tex-text]{TeX Gyre Termes}
\setsansfont[Mapping=tex-text]{TeX Gyre Adventor}
%\setmainfont{TeXGyreTermes}
%\setmainfont{DejaVu Serif}
%\setmainfont{Bitstream Vera Serif}
%\setmonofont{TeX Gyre Cursor}
\setmonofont{DejaVu Sans Mono}
\usepackage{draftwatermark}

% \usepackage{bibentry,natbib}

\usepackage{graphicx}

\usepackage{hyperref}

\usepackage{soul}

\usepackage{relsize}

\usepackage[style=authoryear,natbib=true]{biblatex}
%\addbibresource{JSB2013.bib,typografia.bib}
%\addbibresource{4JSB2014.bib}
\AtEveryBibitem{\clearfield{note}}



\newcommand{\program}[1]{\textsf{#1}}

\title{U4PL tools}
\author{Janusz S. Bień}

\date{\today}


\begin{document}
\maketitle
% \pagestyle{empty}

% no math
\catcode`\&=12
\catcode`\_=12

\section{Unicode Character Database}
\label{sec:unic-char-datab}

On Debian the tools use by default DCB provided by the package
\texttt{unicode-data}. The location of DCB should be however
configurable for portability and for access to earlier versions of the
standards.

\section{Wiki structure}
\label{sec:wiki-structure}

At present small and capital letters are described separately. It is
not clear how to represent conveniently the information common to both
cases.

\section{U4PL file indexer}
\label{sec:u4pl-file-indexer}

The program makes the list of the \texttt{rst} files in the
subdirectory \texttt{codes} and \texttt{PUA} with the names starting
with \texttt{U+} (\texttt{codes}) or \texttt{M+} (\texttt{PUA}).

The program makes also the list of the entries in the
\texttt{Home.rst} file in the current directory. The entries have the form
of hyperlinks such as

{\relsize{-2}
\begin{verbatim}
`'LATIN CAPITAL LETTER O WITH STROKE' (U+00D8) <codes/U+00D8_LATIN_CAPITAL_LETTER_O_WITH_STROKE>`_
\end{verbatim}
}

If an entry is missing for a file, the program adds it on the end of the list.

If the file is missing for an entry, the program comments the entry
out and write a warning message.

If the program is called with an argument, it performs the operations
only for the specified file.

\section{U4PL template creator}
\label{sec:u4pl-templ-creat}

The program takes a hexadecimal code point as an argument. It creates
a template file for it and, if appropriate, the template for the
opposite case with the appropriate hyperlinks.

For example, for \texttt{0253} it creates the file
\texttt{U+0253_LATIN_SMALL_LETTER_B_WITH_HOOK.rst} containing in
particular:

\begin{itemize}
\item \texttt{`'LATIN SMALL LETTER B WITH HOOK' (U+0253) <http://www.fileformat.info/info/unicode/char/0253/index.htm>`_
}
\item \texttt{Unicode 1.1.0 (June, 1993)}
from \texttt{DerivedAge.txt}
\item 
  \begin{quote}
    \obeylines
    	* implosive bilabial stop
	* Pan-Nigerian alphabet
	* uppercase is 0181
  \end{quote}
from \texttt{NamesList.txt}
\item Uppre case \texttt{U+0181} from \texttt{UnicodeData.txt}
\item Wikpedia link \url{http://en.wikipedia.org/wiki/%C9%93} (the last element is in UTF-8, the address may be invalid).
\item DecodeUnicode link \url{http://www.decodeunicode.org/u+0253}
\item The character itself: \texttt{Local font: ɓ}


\end{itemize}

\end{document}

/usr/share/doc/unicode-data

/usr/share/unicode

%%% Local Variables:
%%% mode: latex
%%% TeX-master: t
%%% TeX-engine: xetex
%%% TeX-PDF-mode: t
%%% coding: utf-8
%%% End:
